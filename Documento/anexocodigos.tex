\chapter*{Códigos}
\addcontentsline{toc}{chapter}{Anexo: Códigos de las distintas simulaciones}
A continuación se muestran los códigos del algoritmo de Gillespie y de la resolución de la ecuación de Langevin.
%\lstset{language=Python, breaklines=true, basicstyle=\footnotesize}

\begin{lstlisting}[frame=single,caption=Algoritmo de Gillespie.]
        def gillespie_SIS(infected, recovery_rate, infecting_rate, tmax):
        time = np.array([0.])
        inf = np.array([infected])
        t = 0
        while t < tmax and inf[-1] > 0:     
            r1 = infecting_rate * infected * (individuos-infected) / individuos 
            r2 = recovery_rate * infected   
            r_total = r1 + r2
            delta_t = np.random.exponential(scale=1/r_total)
            prob = np.random.rand() 
            if prob < r1/r_total: 
                infected += 1
            else: 
                infected -= 1
                if infected == 0:
                    infected +=1
            t += delta_t
            time=np.append(time,t)
            inf=np.append(inf,infected)
        return time, inf
\end{lstlisting}
\newpage
\begin{lstlisting}[frame=single,caption=Resolución de la ecuación de Langevin.]
    import numpy as np
    import matplotlib.pyplot as plot
    
    def langevin(deltat,x_0,individuos,infecting_rate,recovery_rate,tmax):
        t=0.
        time = np.array([0.])
        x = np.array([x_0])
        sqrtdeltat=np.sqrt(deltat)
        while t < tmax:
            f = x[-1] + deltat*(infecting_rate*x[-1]*(1-x[-1])-recovery_rate*x[-1])+sqrtdeltat*np.sqrt((infecting_rate*x[-1]*(1-x[-1])+recovery_rate*x[-1])/individuos)*np.random.normal()
            if f<0:
                f=1
            x = np.append(x,f)
            t += deltat
            time = np.append(time,t)
        return x, time
\end{lstlisting}
\newpage
Con el siguiente código podremos obtener los puntos para la representación de los diagramas de fases con y sin barras de error para el número medio de infectados y el diagrama de fase
para la desviación estándar. De la misma forma podremos obtenerlo utilizando la función langevin.
\begin{lstlisting}[frame=single]
    puntos = 100
    for j in range (3):
        medias=np.array([]) #Vector que contendrá las medias
        inf_rates=np.array([])#Vector que contendrá los lambdas
        desviacion=np.array([])
        individuos = 10**(j+1)
        infectados_inicio = individuos/10
        tmax = 100
        for i in range(puntos):
            time, inf= gillespie_SIS(infectados_inicio, recovery_rate, 0.+0.125*i,tmax)
            inf_est=np.array(inf[int(np.ceil(np.size(inf)/2)):np.size(inf):1])
            medias=np.append(medias,[np.mean(inf_est)/individuos]) 
            desviacion=np.append(desviacion,[np.std(inf_est)/individuos])
            inf_rates=np.append(inf_rates,0.+0.125*i)
            print(np.size(inf_rates))
    plot.plot(inf_rates,medias,'-', label='N=%d'%(individuos))
    plot.plot(inf_rates,desviacion,'-', label='N=%d'%(individuos)) 
    plot.errorbar(inf_rates, medias, yerr=desviacion, label='N=%d'%(individuos), fmt = '-', ecolor='k', capsize=3, capthick=1)   
\end{lstlisting}    
\newpage
Con el siguiente código podremos obtener los datos las trayectorias temporales para diferentes condiciones iniciales y las series temporales. 
De igual manera que anteriormente, se pueden obtener utilizando la función gillespie$\_$SIS

\begin{lstlisting}[frame=single]
deltat=0.1
x_0=0.1
individuos=1000
infecting_rate=2
recovery_rate=1
tmax=50
trayectorias=5
#Condiciones iniciales
for j in range(trayectorias) :
        x_0=0.1+0.2*j
        x , time = langevin(deltat,x_0,individuos,infecting_rate,recovery_rate,tmax)
        plot.plot(time,x,'-',label=f"$x_0= %.2f$"%(x_0))

#Series temporales
for j in range(trayectorias) :
        infecting_rate=0.5+0.5*j
        x , time = langevin(deltat,x_0,individuos,infecting_rate,recovery_rate,tmax)
        plot.plot(time,x,'-',label=f"$ \lambda=%.2f$"%(infecting_rate))
\end{lstlisting}  

