\chapter{Objetivos}\label{ch:Objetivos}
En este Trabajo Fin de Grado se han planteado los siguientes objetivos:
\begin{itemize}
    \item Estudiar el modelo susceptible-infectado-susceptible (SIS) en un sistema de tamaño finito y    por tanto, sujeto a fluctuaciones. Para ello, haremos uso de la teoría de procesos estocásticos
    \item Utilizar métodos Montecarlo, la ecuación de Langevin y el algoritmo de Gillespie para estudiar la evolución del sistema.
    \item Implementar en lenguaje Python dicha ecuacione y algoritmo para su resolución.
    \item Visualizar y analizar los resultados obtenidos para distintos órdenes de población e interacción. En concreto, nos centraremos en el estudio del diagrama de fase y la amplitud de las fluctuaciones.
\end{itemize}