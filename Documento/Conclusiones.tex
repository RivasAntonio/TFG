\chapter{Conclusiones}

Con la realización de este trabajo hemos podido iniciarnos en la teoría de procesos estocásticos, teoría fundamental en gran cantidad de ramas
de la física. Hemos podido comprobar como podemos encontrar y/o aplicar la física en otras ramas del conocimiento, como es el caso de este trabajo
y la epidemiología. 

El uso de modelos matemáticos nos ha permitido caracterizar el comportamiento de las epidemias, algo imprescindible para evitar 
o minimizar los daños que estas causan en la sociedad. Para la resolución de las diferentes ecuaciones del modelo, como en la 
mayoría de problemas actuales; es necesario el conocimiento de lenguajes de programación, ya que cuando consideramos situaciones no 
ideales, la mayoría de problemas no tienen solución analítica. 

Como punto de estudio principal, hemos estudiado los diagramas de fase bajo el modelo elegido. Obteniéndo así gran cantidad de información
para distintos valores de los parámetros como lo son el tamaño de la población, las condiciones iniciales o las tasas de interacción. Esto
nos permitirá tomar medidas con suficiente antelación para disminuir lo máximo posible los efectos de epidemias que 
podamos modelar con las condiciones estudiadas.

Para finalizar, haremos una pequeña discusión sobre los dos métodos utilizados para obtener las trayectorias. Los resultados obtenidos son muy similares a los esperados, por lo que ambos métodos
reproducen bien lo esperado, pero es interesante hacer una pequeña discusión sobre las ventajas e inconvenientes de sendos métodos.

Por una parte, con el algoritmo de Gillespie no realizamos ninguna aproximación, por lo que los resultados obtenidos serán 
buenos para cualquier valor de los diferentes parámetros de nuestro sistema. Un inconveniente de este método es el tiempo de cálculo.
Para poblaciones de órdenes mayores a los mil individuos, el tiempo de simulación aumenta considerablemente. Además, para altas 
tasas de infección también aumenta el tiempo de simulación con el algoritmo de Gillespie.

Por otra parte, en el caso de la ecuación de Langevin tenemos la restricción de que debemos tener una población grande para poder aplicar
la aproximación. Sin embargo, para poblaciones mayores al centenar de individuos esta aproximación es asumible, por lo que para la gran mayoría 
de casos de interés de nuestro modelo podemos aplicar esta aproximación. La principal ventaja de este método es el tiempo de simulación,
el cual es prácticamente fijo, ya que nosotros elegimos el paso temporal para la resolución de la ecuación diferencial, por lo que al contrario
del algoritmo de Gillespie, es independiente al tamaño de la población y de la tasa de infección.

En general, como se ha comentado, ambos son buenos métodos, pero debido a que en el caso que nos concierne, el estudio de las epidemias, es mejor
utilizar la ecuación de Langevin, ya que se reduce considerablemente el tiempo de simulación obteniendo aun así resultados igual de satisfactorios 
que con el algoritmo de Gillespie.


\newpage
\chapter*{Conclusions}
With the completion of this work, we have been able to delve into the theory of stochastic processes, a fundamental theory 
in many branches of physics. We have verified how we can find and/or apply physics in other fields of knowledge, such as this 
work and epidemiology.

The use of mathematical models has allowed us to characterize the behavior of epidemics, something essential to 
prevent or minimize the damages they cause in society. To solve the different equations of the model, as in most current 
problems, knowledge of programming languages is necessary, because when we consider non-ideal situations, the majority of 
problems do not have an analytical solution.

As the main focus of study, we have examined phase diagrams, obtaining a wealth of information for 
different parameter values such as population size, initial conditions, or interaction rates. This will enable us to take 
actions to minimize the effects of epidemics that we can model with the studied conditions.

Finally, we will have a brief discussion on the two methods used to obtain the trajectories. The results obtained are very 
similar to the expected ones, so both methods accurately reproduce the expected outcomes. However, it is interesting to discuss 
the advantages and disadvantages of each method.

On one hand, with the Gillespie algorithm, we do not make any approximations, so the results obtained will be good for any 
value of the different parameters in our system. One drawback of this method is the computational time. For populations larger 
than a thousand individuals, the simulation time increases considerably. Additionally, for high infection rates, the simulation 
time also increases with the Gillespie algorithm.

On the other hand, in the case of the Langevin equation, we have the restriction that we need a large population to apply 
the approximation. However, for populations larger than a hundred individuals, this approximation is reasonable, so for the 
vast majority of cases of interest in our model, we can apply this approximation. The main advantage of this method is the 
simulation time, which is practically fixed since we choose the time step for the resolution of the differential equation. 
Thus, unlike the Gillespie algorithm, it is independent of the population size and infection rate.
\newpage
In general, as mentioned, both methods are good, but due to the specific nature of our study of epidemics, it is better 
to use the Langevin equation. This significantly reduces the simulation time while still obtaining equally satisfactory 
results as with the Gillespie algorithm.